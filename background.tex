\tab For many users, the primary use of a social media account is to share personal news and updates, as well as links to articles and websites they have an interest in or support. Most of these types of posts encourage feedback in the form of "likes" or "pluses" and comments that can be left by friends of the poster. This is where privacy risks can arise when a user posts without any privacy set at all (globally visible to anyone), leaving the contents of the post and the comments readable by anyone who visits the social media page. The user or commenters may inadvertently reveal information about themselves or have this information picked up by a stalker or harasser. Small bits of information about a person may not be inherently harmful on their own but compiled together over time into a larger profile, may create a more complete picture that can be used to cross-reference and accurately identify a target.\cite{ghazinour2016yourprivacyprotector} 
\\
\tab Users may also have a large cross-section of friends and family they stay connected with through social media, with a large gap of beliefs, culture and backgrounds. It may not be reasonable or wise to allow all connections on a friends network to have read/write access to a post that may be controversial, nor would it be reasonable to expect a user to refrain from posting on topics he/she is interested in solely to cater to a few people on their list that might take offense or cause a disruption. Many social media accounts allow the creation and use of {\it filters} which users can customize by adding (or removing) friends to and using these as custom privacy settings which can be applied on a per-post basis. This way, the user is no longer constrained to the types or subjects of posts he/she would like to share, and they can ensure the correct audience is exposed to the subject, minimizing potential privacy issues and offensive/disruptive content.
\\
\tab One of the inherent problems with social media though is the lack of emphasis placed on security and privacy, as one of the goals of these types of applications is sharing content and connecting people. Increased privacy and being less visible globally actually works against the connectivity model. With over 1.7 billion users on Facebook\cite{statista}, high security, high privacy options should be standard, with opt-out being an active action the user has to set instead of having to opt-in and set their privacy settings on or to higher levels. Setting up filters aren't always clearly defined and easy to use and either default to the last one used or have to be set each time a post is made. This can introduce human error, as it relies on the user to be vigilant and cognizant each time they make a post to check the privacy settings and ensure they are correct at that particular moment. 
\\
\tab Our hope in a Chrome browser extension that can analyze the contents and flag the content for potential privacy leaks and controversial topics before the content is posted to Facebook is that it would lower the risk of human error and forgetfulness and allow users to post more freely without having to worry about whether they've set the proper privacy/security permissions or not. 
