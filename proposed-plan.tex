\tab The overall goal of this research is to establish a mechanized way to intelligently protect the dissemination of a user's Facebook posts to a specific group of friends that is visible and transparent to the user in a simple and easy to understand fashion. Through the use of our application the user will be alleviated from having to tediously choose what friends a given post will be visible to and can instead follow the recommendations provided to them by the app. The application will provide a clean and easy to read interface that gives full transparency to the user on who his/her posts will be visible to. This functionality will help keep the posted content of a user confined to parties that have been verified as trustworthy for the given category of the content. 
\\
\tab The second goal of this research is to provide a clear recommendation for users to use when setting up Friend Lists in Facebook and to provide them with the tools needed to use these lists as personal privacy security zones. The goal here is to educate users on how to be more conscientious of their social information and how to prevent privacy leaks into the social web. This will accomplished through the startup rules that will be laid out in the guidelines for creating Friend Lists (security zones) and how to best use these lists when posting content. 
\\
\tab The two goals above have been presented separately but in all actuality they are closely related. The efficacy of the primary goal is very much dependent on the successful setup of meaningful and accurate Friend Lists. That is why making the manual part of our research process easy to implement by the user is important. We would like to automate this process but due to restrictions on the information that is publicly available from the Facebook Graph APIs this is not feasible at the current time. The guidelines that we will lay out could later be considered for developing an automated protocol for bucketing users into security zones. 
