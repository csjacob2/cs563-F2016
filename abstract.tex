\tab With the huge growth in popularity of social media accounts and applications, privacy and security of users' personal information and the content they post publicly exposes them to a wide unknown audience which can potentially cause serious problems in their personal and professional lives. Lack of understanding how privacy settings work or carelessness (caused by human error) has led to severe repercussions and damages to thousands of people. We propose a solution through a Chrome extension for Facebook that will allow users to define sets of rules to categorize people in their Friends List into smaller groups in order to allocate personalized privacy settings. This extension would then analyze the contents of a post, prior to being made public, and measure the contents against a rules dictionary and determine if the post requires a security setting (based on the rules pre-defined by the user). The user can then decide if they want to apply the security/privacy filter or allow it to be posted without. Alternatively, the extension could automatically apply the security/privacy filter without prompting for confirmation if the user chooses to set it up accordingly. The goal of this extension is to lower the risk of human error and increase the privacy and security of the contents of the posts made by a user by monitoring and intervening (or reminding) the user to apply the appropriate filter if the contents are flagged by the extension.