\tab To achieve our goal in helping to protect the scope of a user's posted content on Facebook we propose a two pronged approach. The first prong of this approach is a Facebook application that will intelligently control the audience of a user's post based off of security guidelines set in place by the user. When a user downloads our chrome extension for the first time he/she will be prompted to perform an initial setup. The setup will consist of the user creating different mappings or rather rules that link a given Friend List (e.g. Family) to a given category and action (e.g. politics, do not share). The categories will be predefined to begin with but the application will allow for custom categories to be created by the user later on. The application will be using the Graph API provided by Facebook to grab the valid set of Friend Lists for the user\cite{weaver2013facebook}.
\\
\tab Once a user has established an initial set of rules they will then be presented with a dialog box. That dialog box will allow them to type in the contents of a post. It will also display the set of Friend Lists that the post will be visible to, a button to validate that list, and a button to post the content. When the user clicks 'post' the content will be published out to the shown Friend Lists. That list will be generated by our algorithm upon the user clicking the validate button. The algorithm to generate that list consists of two main parts: a weight function that assigns categories and their given weights to the given content and a decision function that takes the assigned categories and the user's rules to spit out a recommend set of Friend Lists. Here the categories for a post will be assigned using a dictionary that will be pre-loaded with words that are mapped to a given category.
\\
\tab The second prong is a set of guidelines that will outline best practices for the user to take when setting up security zones in Facebook through the use of Friend Lists. This will take the form of a set of rules to be given to the user that will be used to assess the risk of a given friend based off of the information disclosed on the friend's profile\cite{ghazinour2016yourprivacyprotector}. Facebook no longer allows this information to be collected through the APIs; therefore this piece of our research has become a more manual process for the user. The guidelines will also include suggestions for different over-arching social privacy zones that are clearly defined and will get the user up and running quickly. The Friend Lists defined here will then be used by the application described above when a user is requesting advice for the specific Friend Lists a post should be sent to. 
